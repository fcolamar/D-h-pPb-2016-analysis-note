The data samples used for the analyses were the FAST and CENT$\_$woSDD samples from periods LHC16q and LHC16t (merged AOD samples). The reason for this choice is explained later on. It was verified, by looking at D-meson and track $\eta$ and $\phi$ distributions, and at the mixed-event correlation distributions for each subsamples, that no visible differences arose for the four periods, hence it was possible to perform the analysis directly on the merged samples without any bias.

The Monte Carlo productions adopted for this study were:
 \begin{enumerate}
 \item LHC17d2a$\_$fast$\_$new, a HIJING production with enrichment, for each event, of c OR b quarks and their decay chains, performed by PYTHIA6 with Perugia2011 tune, and with forced hadronic decays of the charmed hadrons. This production was used for D-meson efficiency evaluation, purity estimation and Monte Carlo closure test.
 \item LHC17a2b$\_$cent$\_$woSDD and LHC17a2b$\_$fast, minimum-bias productions perfomed with DPMJET generator, used for the evaluation of the tracking efficiency.
\end{enumerate}

Table 1 shows the list of runs used for the analysis, for each of the data taking periods, and of the Monte Carlo productions used to evaluate the corrections:

AFTER THE TABLE, SOME LINES FOR THE EVENT SELECTION HAVE TO BE ADDED (trigger class and mask, pile up cuts, physics selection, full 0-100 centrality range with ZNA estimator, zvtz $<$ 10 cm, etc.)

FINALLY, A DISCUSSION FOR wsdd VS wosdd SAMPLES MUST BE DONE, WITH THE REASONING OF OUR FINAL CHOICE (WITH PLOTS FOR AT LEAST ONE MESON, BUT BETTER BOTH D+ and D*)

 \vspace{10 mm}

% Systematic table
\begin{table}[h]
\begin{tabular}{ p{1.2cm} | p{4.2cm} |  p{7cm} |  p{1.2cm}}
{\normalsize \textbf {Type}} &       {\normalsize \textbf {Production}} &       {\normalsize \textbf {Run list}} & {\normalsize \textbf {nEvents}} \\
\\ \hline
Monte-Carlo & LHC17d2a$\_$fast$\_$new/AOD (c/b enriched), LHC17a2b$\_$fast (MB), LHC17a2b$\_$cent$\_$woSDD (MB) &265525, 265521, 265501, 265500, 265499, 265435, 265427, 265426, 265425, 265424, 265422, 265421, 265420, 265419, 265388, 265387, 265385, 265384, 265383, 265381, 265378, 265377, 265344, 265343, 265342, 265339, 265338, 265336, 265335, 265334, 265332, 265309, 267165, 267164, 267163, 267166 = \textbf{[22 runs]} & ??M\\
\\ \hline

\multirow{7}{*}{} Data&``LHC16q, pass1$\_$CENT$\_$woSDD/AOD" & 265525, 265521, 265501, 265500, 265499, 265435, 265427, 265426, 265425, 265424, 265422, 265421, 265420, 265419, 265388, 265387, 265385, 265384, 265383, 265381, 265378, 265377, 265344, 265343, 265342, 265339, 265338, 265336, 265335, 265334, 265332, 265309 = \textbf{[32 runs]}& 261M total\\
                  &``LHC16q, pass1$\_$FAST/AOD" &265525, 265521, 265501, 265500, 265499, 265435, 265427, 265426, 265425, 265424, 265422, 265421, 265420, 265419, 265388, 265387, 265385, 265384, 265383, 265381, 265378, 265377, 265344, 265343, 265342, 265339, 265338, 265336, 265335, 265334, 265332, 265309 = \textbf{[32 runs]} & 260M \\
% \\ \hline
 & ``LHC16t, pass1$\_$CENT$\_$woSDD/AOD" & 267166, 267165, 267164, 267163 = \textbf{[4 runs]} & 40M \\
% \\ \hline
  & ``LHC16t, pass1$\_$FAST/AOD" & 267166, 267165, 267164, 267163 = \textbf{[4 runs]} & 41M \\
 \hline \hline
\end{tabular}
\\
\caption {Data Set and Run list}
\end{table} 