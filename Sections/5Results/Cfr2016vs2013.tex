In Figure \ref{fig:CfrAverage}, the average correlation distributions from the published analysis in p-Pb 2013 sample (black points) and the new p-Pb 2016 sample (red points), both at 5 TeV, are compared. As it's evident, the statistical and systematic uncertainties are largely reduced in the new data sample. The feature of the correlation distributions are the same in both systems, and an overall compatibility of the points is observed. Only in the near-side region, the 2016 data points have a tendency of being slightly below the 2013 data points. In part, this can be partially explained with the different procedure for assessing the B to D decay topology bias (2016 data corrected, with a slight downward shift for the first two points, while for 2013 data only a downward systematic uncertainty was applied.

\begin{figure}[!htbp]
\centering
%Marianna
\centering
{\includegraphics[width=0.47\linewidth]{figures/Cfr2013vs2016/Average_Cfr_2013_2016_Pt5to8_Thr03to99.png}}
{\includegraphics[width=0.47\linewidth]{figures/Cfr2013vs2016/Average_Cfr_2013_2016_Pt5to8_Thr03to1.png}}
{\includegraphics[width=0.47\linewidth]{figures/Cfr2013vs2016/Average_Cfr_2013_2016_Pt5to8_Thr1to99.png}}
{\includegraphics[width=0.47\linewidth]{figures/Cfr2013vs2016/Average_Cfr_2013_2016_Pt8to16_Thr03to99.png}}
{\includegraphics[width=0.47\linewidth]{figures/Cfr2013vs2016/Average_Cfr_2013_2016_Pt8to16_Thr03to1.png}}
{\includegraphics[width=0.47\linewidth]{figures/Cfr2013vs2016/Average_Cfr_2013_2016_Pt8to16_Thr1to99.png}}
\caption{Comparison of 2016 (red) and 2013 (black) results for azimuthal correlation distributions, for the common $\pt$ ranges.}
\label{fig:CfrAverage}
\end{figure}

Figure \ref{fig:CfrObs} shows the same comparison for the fit observables. Also in this case the uncertainties are largely reduced for the 2016 analysis. While the away side features are compatible (but with large uncertainties) and the near-side widths are on top of each other, for the near-side yields a slight decrease of the 2016 results is observed (though well within the uncertainty). This is a direct consequence of the feature just observed in the comparison of the near-side peak point of the azimuthal correlation distributions (i.e. that the 2016 data points are slightly lower than the 2013 ones).

\begin{figure}[!htbp]
\centering
%Marianna
\centering
{\includegraphics[width=0.31\linewidth]{figures/Cfr2013vs2016/NSYield_Cfr_2013_2016_Thr03to99.png}}
{\includegraphics[width=0.31\linewidth]{figures/Cfr2013vs2016/NSYield_Cfr_2013_2016_Thr03to1.png}}
{\includegraphics[width=0.31\linewidth]{figures/Cfr2013vs2016/NSYield_Cfr_2013_2016_Thr1to99.png}}
{\includegraphics[width=0.31\linewidth]{figures/Cfr2013vs2016/NSsigma_Cfr_2013_2016_Thr03to99.png}}
{\includegraphics[width=0.31\linewidth]{figures/Cfr2013vs2016/NSsigma_Cfr_2013_2016_Thr03to1.png}}
{\includegraphics[width=0.31\linewidth]{figures/Cfr2013vs2016/NSsigma_Cfr_2013_2016_Thr1to99.png}}
{\includegraphics[width=0.31\linewidth]{figures/Cfr2013vs2016/ASYield_Cfr_2013_2016_Thr03to99.png}}
{\includegraphics[width=0.31\linewidth]{figures/Cfr2013vs2016/ASYield_Cfr_2013_2016_Thr03to1.png}}
{\includegraphics[width=0.31\linewidth]{figures/Cfr2013vs2016/ASYield_Cfr_2013_2016_Thr1to99.png}}
{\includegraphics[width=0.31\linewidth]{figures/Cfr2013vs2016/ASsigma_Cfr_2013_2016_Thr03to99.png}}
{\includegraphics[width=0.31\linewidth]{figures/Cfr2013vs2016/ASsigma_Cfr_2013_2016_Thr03to1.png}}
{\includegraphics[width=0.31\linewidth]{figures/Cfr2013vs2016/ASsigma_Cfr_2013_2016_Thr1to99.png}}
{\includegraphics[width=0.31\linewidth]{figures/Cfr2013vs2016/Pedestal_Cfr_2013_2016_Thr03to99.png}}
{\includegraphics[width=0.31\linewidth]{figures/Cfr2013vs2016/Pedestal_Cfr_2013_2016_Thr03to1.png}}
{\includegraphics[width=0.31\linewidth]{figures/Cfr2013vs2016/Pedestal_Cfr_2013_2016_Thr1to99.png}}
\caption{Comparison of the average D-h azimuthal correlation properties between 2016 p-Pb (red) and 2013 p-Pb (black) data analysis, for the common $\pt$ ranges of D meson and associated particles.}
\label{fig:CfrObs}
\end{figure}
