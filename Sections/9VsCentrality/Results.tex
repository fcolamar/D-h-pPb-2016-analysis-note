The procedure for, after having evaluated all the needed correction, is totally equivalent to that described in Section 5.
As a quick reminder, the weighted average of the azimuthal correlation distributions for the three D-meson species are evaluated for each kinematic range under study, and a fit function (composed of a double Gaussian for the near- and away-side peak and a constant baseline) is applied to the average distribution, in order to extract quantitative observables as the NS and AS associated yield and widths.
Of course, the procedure is performed independently on each of the three centrality classes, and a comparison of the correlation distributions (after the baseline subtraction) and of the NS and AS peak observables is done.
The results of the analysis for each centrality class, and of the comparison, are shown in the following.

\subsubsection{Single-meson azimuthal correlation distributions}
- mostrare mesoni singoli da script pre-average

\subsubsection{Average-D correlation distributions}
- come da titolo

\subsubsection{Observables from fit to correlation distributions}


In the following:
\textit{\textbf{SHOW FIT OBSERVABLES AND THE PLOTS OF THEIR SYST UNCERTAINTIES!}}

\subsubsection{Plots proposed for preliminaries}
- mostra quelli da approvare 